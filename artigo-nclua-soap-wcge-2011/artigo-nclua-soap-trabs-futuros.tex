Como trabalhos futuros, pretende-se: concluir a implementação do \textit{parse} automático do 
documento WSDL e geração de \textit{stubs} Lua, contendo funções \textit{proxies} para realizar a chamada aos métodos remotos
(semelhante a ferramentas como o \textit{wsdl2java}\footnote{\url{http://ws.apache.org/axis/java/user-guide.html}});
incluir tratamento de exceções para permitir que as aplicações
de TVDi possam emitir mensagens amigáveis ao usuário quando uma requisição HTTP falhar;
e implementar o envio de anexos em formato \textit{Multipurpose Internet Mail Extensions} (MIME)
\cite{rfc2045}\nomenclature{MIME}{\textit{Multipurpose Internet Mail Extensions}}.

No módulo NCLua HTTP, pretende-se realizar testes
de conformidade utilizando-se os métodos HTTP \textit{OPTIONS, HEAD, PUT} e \textit{DELETE}. Pretende-se também
implementar mais funcionalidades no módulo, como realizar redirecionamentos automaticamente
a partir de respostas HTTP 301 e 302, e implementar recursos de conexões persistentes.


