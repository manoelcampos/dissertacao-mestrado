Os módulos implementados facilitam a convergência entre \textit{Web} e TV, escondendo
os detalhes de implementação dos protocolos HTTP e SOAP do desenvolvedor
de aplicações de TVDi, permitindo o surgimento de novas aplicações interativas 
que fazem uso de conteúdo da \textit{Internet}.

\added{
As implementações apresentadas trazem ainda como benefícios
para os desenvolvedores de aplicações NCLua:
\begin{itemize}
	\item encapsulamento de toda a complexidade dos protocolos HTTP e SOAP;
	\item encapsulamento dos detalhes do procotolo TCP do sub-sistema Ginga-NCL;
	\item facilidade e agilidade no desenvolvimento de aplicações com interatividade plena;
	\item serem implementações de código aberto que permitem a evolução das mesmas pela comunidade de desenvolvedores
	de aplicações de TV Digital.
\end{itemize}
}

Algumas aplicações foram desenvolvidas como prova de conceito do uso dos módulos. Algumas
delas foram: NCLua RSS Reader (leitor de RSS); Enquete TVD; TVD Quizz; NCLua Tweet (cliente de \textit{Twitter}); 
Rastreador de Encomendas dos Correios.
O NCLua SOAP tem permitido, recentemente, o desenvolvimento de aplicações tais como: \textit{T-Government}
\cite{tgov2010barbosa}, recomendação de conteúdo \cite{gatto2010BIPODiTVR}, entre outras. 

Atualmente, o processo de obtenção dos dados para realizar a chamada a um método remoto em um \textit{WS}
ainda é praticamente todo manual, no entanto, 
após o desenvolvedor obter tais dados para o método remoto desejado, a realização da requisição é bem simples.
O \textit{feedback} dos usuários tem mostrado que o módulo está sendo bastante útil para 
a comunidade, além de permitir a evolução do mesmo.

Uma tabela comparativa e análise de desempenho dos módulos foram construídas, no entanto, não
foram apresentadas aqui por restrições de espaço.
