Como trabalhos futuros, pretende-se concluir a implementação do \textit{parse} automático do 
documento WSDL e geração de \textit{stubs} Lua, contendo funções \textit{proxies} para realizar a chamada aos métodos remotos
(semelhante a ferramentas como o \textit{wsdl2java}\footnote{http://ws.apache.org/axis/java/user-guide.html}, 
pretende-se transformar o \textit{script} \textit{wsdlparser} em um \textit{wsdl2nclua})
e incluir tratamento de exceções para permitir que as aplicações
de TVDi possam emitir mensagens amigáveis ao usuário quando uma requisição HTTP falhar.

O módulo NCLua HTTP permite que seja utilizado qualquer método HTTP em uma requisição, no entanto,
apenas os método \textit{GET} e \textit{POST} foram testados. Desta forma, pretende-se realizar testes
de conformidade utilizando-se os métodos HTTP \textit{OPTIONS, HEAD, PUT} e \textit{DELETE}. Pretende-se também
implementar mais funcionalidades no módulo, como realizar redirecionamentos automaticamente
a partir de respostas HTTP como 301 e 302, além de implementar alguns recursos do HTTP/1.1, como
conexões persistentes.

