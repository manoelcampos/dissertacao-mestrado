\section{Decisões de Projeto}

Para o desenvolvimento do projeto, optou-se por seguir o padrão
de módulos, amplamente utilizado na atual versão da linguagem Lua.
Um módulo encapsula funções com objetivos correlatos
e tal padrão é bastante conhecido dos programadores Lua.

Tal modelo segue o paradigma de programação estruturada, assim,
um módulo consiste de um conjunto de funções que são chamadas
pelo desenvolvedor que for utilizá-lo.

Na geração das requisições, optou-se por fazer o \textit{(un)marshalling}
de XML para tabelas Lua por estas serem as estruturas de dados
padrões disponibilizadas pela linguagem e por serem estruturas
bastante flexíveis e fáceis de manipular, devendo ser de conhecimento
de qualquer programador Lua.

Como o conteúdo das requisições HTTP e SOAP é apenas texto, formatado
segundo o padrão de cada protocolo, a geração e formatação
deste conteúdo foi bastante simples e direta. Devido a
\textit{strings} em Lua serem imutáveis\cite{ierusalimschy2006programming},
para economizar memória na concatenação das \textit{strings} que compõe
cada requisição, as tabelas de Lua foram utilizadas
como um \textit{buffer} de \textit{strings} para otimizar o uso de memória RAM.

Na obtenção dos resultados, procurou-se facilitar ao máximo
tal tarefa para o desenvolvedor, permitindo que ele
tenha acesso direto aos dados retornados, sem precisar
conhecer o caminho dentro do XML de retorno onde
a resposta está armazenada, assim como fazem outros \textit{toolkits} SOAP
como a API JAX-WS.

Quanto ao \textit{parser} XML escolhido, não haviam muitas opções
implementadas inteiramente em Lua. 
Todas as implementações testadas foram encontradas
em http://lua-users.org e em \cite{ierusalimschy2006programming}.
Optou-se pelo uso do LuaXML\footnote{http://lua-users.org/wiki/LuaXml},
pois foi a implementação que fazia o \textit{unmarshalling} de XML
para tabelas Lua com a estrutura mais simples e fácil de manipular.
Além do mais, as outras implementações encontradas não funcionaram
adequadamente para XML's mais complexos retornados
por alguns \textit{Web Services}.

\section{Ambiente de desenvolvimento}

Para o desenvolvimento do projeto foi utilizado:
\begin{itemize}
	\item sistema operacional GNU/Linux Ubuntu 10.10 como sistema desktop para realização das tarefas de desenvolvimento:
  \begin{itemize}
  		\item ferramentas \textit{telnet} e \textit{curl} para teste de envio de requisições HTTP/SOAP geradas com os módulos implementados.
  	\end{itemize}	
  \item \textit{soapUI} 3.5.1 para testes de consumo de \textit{Web Services} SOAP;
  \item \textit{Astah Community} 6.1 (antigo \textit{Jude}) para modelagem UML;
	\item IDE Eclipse 3.6:
	\begin{itemize}
		\item \textit{plugin} NCLEclipse 1.5.1;
		\item \textit{plugin} LuaEclipse 1.3.1.
	\end{itemize}
	\item ferramenta LuaDoc\footnote{http://luadoc.luaforge.net} para geração de documentação;
	\item implementação de referência do Ginga-NCL (Ginga \textit{Virtual Set-top Box} 0.11.2):
	\begin{itemize}
		\item módulo \textit{tcp.lua}\footnote{http://www.lua.inf.puc-rio.br/~francisco/nclua/} para envio de requisições TCP;
		\item módulo LuaXML para tratamento de XML;
		\item módulo \textit{LuaProfiler}\footnote{http://luaprofiler.luaforge.net}
	para realização das análises de desempenho, descritas na Seção \ref{sec:analise}.
	\end{itemize}
	\item interpretador Lua para execução do \textit{script wsdlparser.lua} fora do Ginga \textit{Virtual Set-top Box}.
\end{itemize}

\section{Análises de Desempenho do NCLua SOAP} \label{sec:analise}

Utilizando-se a ferramenta \textit{LuaProfiler} (que precisou ter seu processo de compilação alterado para funcionar 
em aplicações NCLua\footnote{http://goo.gl/42CQA}) foram feitas algumas análises
do desempenho do NCLua SOAP para avaliar o tempo gasto para geração da requisição SOAP.
Por meio da função \textit{collectgarbage} da biblioteca padrão de Lua, pôde-se verificar
o total de memória consumida pelo módulo. 

O percentual de uso da CPU foi obtido por meio da ferramenta \textit{top}, existente no sistema
operacional do Ginga \textit{Virtual Set-top Box}. Para a obtenção deste percentual,
executou-se a aplicação NCL (que não inicia o \textit{script} Lua automaticamente) e 
verificou-se o uso da CPU pelo processo de nome "ginga" (responsável
pela execução das aplicações interativas) no momento em que este se tornava
estável. Em seguida, por meio de um comando na aplicação NCL,
o \textit{script} Lua é então executado, gerando e enviando a requisição SOAP.
Após isto, verificou-se o pico de uso do processador, considerando que
os \textit{scripts} Lua testados somente enviam a requisição SOAP e exibem o retorno no terminal.
Com a diferença entre o pico de uso do processador depois da execução do \textit{script}
e o valor observado antes do início do mesmo, obtém-se o percentual de uso do processador
pelo NCLua SOAP.

A Tabela \ref{tab:analise-desempenho} apresenta os resultados obtidos para algumas das aplicações testadas.

\begin{center}
\scriptsize{
	\begin{tabular}{|p{4.5cm}|p{2.8cm}|p{2.8cm}|p{1.2cm}|p{1.2cm}|} %{|l|c|c|}
  \hline
		\textbf{\textit{Web Service} consumido} & 
		\textbf{Parâmetros} \par\textbf{de entrada} &
		\textbf{Tempo de geração da requisição:}\par\textbf{chamada ao método \textit{call}} \textbf{(em segundos)} & 
		\textbf{Uso de Memória RAM}\par\textbf{(em KB)} &
		\textbf{\% de Uso da CPU}\\
  \hline
		Situação do tempo\par http://www.deeptraining.com\par/webservices/weather.asmx & 
		\textit{City} = "Brasília" & 0.13 & 121.78 & 0.3 \\
  \hline
		Conversão de Moeda\par
		http://www.webservicex.net\par/CurrencyConvertor.asmx &
		\textit{FromCurrency} = "USD"\par        
    \textit{ToCurrency} = "BRL" & 
    0.13 & 121.65 & 0.3 \\  
  \hline
		Consulta de endereço a partir do CEP\par
		http://www.maniezo.com.br\par/webservice/soap-server.php &
    cep = "77021682" & 
    0.13 & 133.17 & 0.3 \\  
	\hline
	\end{tabular}
	\captionof{table}{Análise de Desempenho do NCLua SOAP}
	\label{tab:analise-desempenho}
}
\end{center}

Como pode ser observado na Tabela \ref{tab:analise-desempenho},
o tempo para geração da requisição, ou seja, a chamada ao método \textit{call}
do módulo, na implementação de referência do Ginga (Ginga \textit{Virtual Set-top Box}, versão 0.11.2)
está em torno de 0.13 segundo. O método \textit{call} inclui a conversão dos dados passados em uma tabela Lua
para o formato XML para serem enviados ao \textit{Web Service}. Tal método não aguarda 
a conexão ao servidor \textit{Web} e nem o envio e resposta da requisição, pois tais
operações são assíncronas no Ginga-NCL. Assim, o tempo de execução
do método \textit{call} reflete apenas o tempo de geração da requisição SOAP. 
Os tempos obtidos nos testes não variaram muito,
e a variação vai depender apenas do total de parâmetros passados.

O uso de memória RAM também ficou baixo, em torno de 120KB, que varia de acordo com
os parâmetros de entrada e saída do método remoto.

Em todos os exemplos executados, pôde-se observar que o percentual de uso da CPU se 
manteve estável em 0.3\%, sendo um valor consideravelmente baixo.

Foram feitas análises do uso direto do módulo \textit{tcp.lua} para verificar o \textit{overhead}
causado pelo NCLua SOAP, que é uma camada sobre o \textit{tcp.lua}. Nos três exemplos
apresentados na Tabela \ref{tab:analise-desempenho}, o tempo de uso da CPU se manteve também estável e igual
aos valores obtidos com o NCLua SOAP, 0.3\%.
Não foram feitas medidas quanto ao tempo gasto (em segundos) para geração da requisição, pois,
para uso direto do \textit{tcp.lua}, o desenvolvedor precisa passar ao módulo uma \textit{string} já com a requisição
SOAP formatada. Assim, não há um \textit{delay} para geração da requisição pois o \textit{tcp.lua}
recebe a mesma pronta para envio (o que é bastante complicado para o desenvolvedor
gerar manualmente tal requisição HTTP/SOAP, além de ser totalmente inflexível).

Quanto ao consumo de memória RAM, o uso direto do \textit{tcp.lua} permite uma otimização
neste sentido, uma vez que a quantidade de varáveis e estruturas de dados
declaradas é muito menor, considerando que o módulo deve receber a requisição
pré-formatada. Para o primeiro exemplo da Tabela \ref{tab:analise-desempenho},
a aplicação consumiu 76.02 KB de memória RAM, para o segundo consumiu 75.59 KB 
e para o terceiro exemplo consumiu 66.28 KB. Tais valores mostram que o consumo
de memória está em torno de 50\% do consumido pelo NCLua SOAP.

Foram feitas aplicações para consumir os mesmos \textit{Web Services} apresentados
na Tabela \ref{tab:analise-desempenho} utilizando o módulo LuaSOAP. No entanto, 
a aplicação funcionou apenas para o \textit{Web Service} de obtenção de endereço
a partir do CEP, que implementa o SOAP 1.1. Para os outros \textit{Web Services} testados,
que implementam SOAP 1.2, o LuaSOAP não enviou a requisição no formato
esperado e os \textit{Web Services} retornaram mensagens de erro.
O LuaSOAP é um projeto que não tem atualizações desde 2004, assim,
tais problemas eram esperados.
Com o \textit{Web Service} de busca de endereço, a aplicação consumiu 
139.20KB de memória RAM, cerca de 6KB a mais que com o NCLua SOAP.
O tempo de geração da requisição foi de apenas 0.01 segundo, bem inferior
ao do NCLua SOAP, que levou 0.13 segundo. Presume-se que este tempo
reduzido gasto pelo LuaSOAP é devido ao mesmo utilizar módulos escritos 
em C, que por serem compilados, têm este ganho de desempenho.

Quanto ao uso de CPU, a aplicação com o LuaSOAP teve pico de 1.0\%, sendo que com o NCLua SOAP
este percentual foi de apenas 0.3. Presume-se que tal valor elevado é devido
ao fato de uso do \textit{parser} XML \textit{Expat}, que é uma implementação mais
completa que o LuaXML usado no NCLua SOAP.

Com tais valores apresentados anteriormente, considera-se que o NCLua SOAP é bastante eficiente
em questões de tempo de processamento e uso de memória. 
Presume-se que o tempo de processamento pode ser reduzido em
um hardware dedicado como o dos \textit{Set-top Boxes}, pois os testes
foram realizados em uma sistema operacional Linux com várias aplicações
em execução e em uma máquina virtual (cujo desempenho é inferior a de uma máquina real
com finalidades específicas).

\section{Limitações do módulo NCLua SOAP}

A atual versão do NCLua SOAP possui algumas limitações, que entendemos não intereferirem no uso do mesmo:

\begin{itemize}
	\item necessidade de o desenvolvedor saber a versão do protocolo SOAP do serviço que ele deseja consumir;
  \item necessidade de extrair manualmente alguns dados como o \textit{namespace} do serviço;
  \item falta de tratamento de erros retornados pelo protocolo HTTP, o que causará comportamentos
  inesperados na aplicação;
  \item necessidade de saber se o serviço utiliza um arquivo \textit{XML Schema Definition} (XSD) externo,
        para poder indicar isto no momento da chamada ao método remoto;
  \item não permitir o envio de anexos em formato \textit{Multipurpose Internet Mail Extensions} (MIME)\cite{rfc2045} 
  \nomenclature{MIME}{\textit{Multipurpose Internet Mail Extensions}}dentro de uma mensagem SOAP.
\end{itemize}

\section{Comparação do NCLua SOAP com outras implementações}

A Tabela \ref{tab:comparacao-toolkits-soap} apresenta um comparativo entre alguns \textit{toolkits} SOAP
e o NCLua SOAP.

\begin{center}
\scriptsize{
	\begin{tabular}{|p{5.5cm}|p{1cm}|p{1cm}|p{1.5cm}|p{1.2cm}|p{2cm}|} %{|l|c|c|}
  \hline
		\textbf{Características} & 
		\textbf{Axis} &
		\textbf{Axis2} &		
		\textbf{PHP} &		
		\textbf{gSOAP} &		
		\textbf{NCLua SOAP} 
    \\
  \hline
		Linguagem & Java & Java & PHP 5 & C++ & NCLua \\  
  \hline
		SOAP 1.1 & Sim & Sim & Sim & Sim & Sim \\  
  \hline
		SOAP 1.2 & Sim & Sim & Sim & Sim & Sim \\  
  \hline
		SOAP com Anexos & Sim & Sim & Sim & Sim & Ainda não \\  
  \hline
		Geração de código cliente a partir do WSDL & Sim & Sim & Sim & Sim & Ainda não \\  
  \hline
		Suporte para formato \textit{document/literal} & Bom & Bom & Médio & Bom & Bom \\  
  \hline
		Requisitos de \textit{runtime} & JVM & JVM & PHP engine & Nenhum & Ginga-NCL \\  
  \hline
		Documentação & Boa & Pequena & Média & Boa & Média \\  		
	\hline
	\end{tabular}
	\captionof{table}{Comparação entre NCLua SOAP e outros \textit{toolkits} SOAP. Adaptada de \cite{louridas2006soap}}
	\label{tab:comparacao-toolkits-soap}
}
\end{center}
