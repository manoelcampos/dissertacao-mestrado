Nas últimas décadas, observou-se uma evolução gradual das tecnologias da informação, e como desdobramento desse evento, a organização e o comportamento humano \replaced{sofreram}{sofrem} modificações. 
Prova disso são os sistemas de compras pelas redes de computadores, que nos últimos tempos tiveram um crescimento acelerado. 
O Brasil, assim como outros países pelo mundo, possui vários casos de sucesso
de lojas de comércio eletrônico (\textit{E-Commerce})\nomenclature{E-Commerce}{Comércio Eletrônico}\footnote{\url{www.americanas.com.br}, \url{www.submarino.com.br}, \url{www.pontofrio.com.br} e outros}.

Esse fato se deve à comodidade de, a partir de um computador ligado à \textit{Internet},  poder-se realizar compras de bens e serviços.  
Nesse modelo, o usuário tem a facilidade e a liberdade de se usar o tempo que julgar necessário para avaliar o produto que deseja adquirir, além de efetuar pesquisas em várias lojas ao mesmo tempo.

Outro fator importante, está na qualidade do atendimento, devido aos atendentes possuírem em mãos um sistema de informação para prover respostas ágeis e credenciadas. Assim, o usuário economiza tempo e se exime de problemas naturais da organização contemporânea como: as filas, gastos de deslocamento, trânsito e estacionamento.

Além disso, as lojas virtuais possuem sistemas de recomendação de produtos, que com base no perfil e nas aquisições habituais do usuário, oferecem produtos incentivando o usuário a utilizar o \textit{E-Commerce}.  Todos esses benefícios são alguns dos fatores de sucesso
das maiores lojas de \textit{E-Commerce} no Brasil e no mundo.

A Pesquisa sobre o Uso das Tecnologias de Informação
e Comunicação no Brasil/TIC Domicílios em 2009 (última divulgada até a data de finalização desta dissertação)\cite{tic2009},
realizada pelo Centro de Estudos sobre as Tecnologias da Informação e da Comunicação (CETIC.br)
\footnote{Vinculado ao Núcleo de Informação e Coordenação do Ponto BR (NIC.br)}
\nomenclature{NIC.br}{Núcleo de Informação e Coordenação do Ponto BR}
e ao Comitê Gestor da \textit{Internet} no Brasil (CGI.br)\nomenclature{CGI.br}{Comitê Gestor da \textit{Internet} no Brasil}
\nomenclature{CETIC.br}{Centro de Estudos sobre as Tecnologias da Informação e da Comunicação}
mostrou que, de 2008 para 2009, houve um aumento de 8\% na consulta a preços de produtos ou serviços na \textit{Internet},
passando de 44\% para 52\% em todo o Brasil, e que houve um crescimento nas compras \textit{on line} de 3\%,
passando de 16\% para 19\% nacionalmente.
A pesquisa ainda revela que, do total de domicílios pesquisados em 2009,
32\% tinham computador (contra 25\% em 2008) e 25\% tinham acesso à \textit{Internet} (contra 18\% em 2008).

Assim os dados da pesquisa supracitada mostram que, no Brasil, as buscas por produtos e serviços na \textit{Internet} é crescente e que gradualmente mais 
pessoas têm acesso ao comércio eletrônico. 

\deleted{
Apesar desse cenário favorável, temos que ressaltar a existência de diversos perigos nas compras via \textit{Internet}. As questões de segurança são vistas como grande entrave à plena execução dessa modalidade de compras, devido às atividades criminosas que visam ganhar dinheiro ilegalmente dentro do sistema.}

\deleted{
Esses perigos são potencializados com atividades
como as de \textit{phishing}, onde um criminoso
tenta assumir a identidade de uma empresa,
enviando mensagens eletrônicas a usuários
em nome daquela, ou mesmo
clonando o \textit{site} da empresa para tentar
enganar os usuários. Estes, ao acessarem
tais \textit{sites}, pensam estar acessando o \textit{site}
da empresa, passando a fornecer dados pessoais
como \textit{login}, senha e números de cartões de crédito,
que são capturados pelo criminoso.
}

\deleted{
Contudo, as vantagens do comércio eletrônico ainda superam os riscos e é importante ressaltar que parte de tais problemas poderia ser resolvida com atitudes preventivas do próprio usuário, como apresenta a Cartilha de Segurança
para \textit{Internet}%\footnote{\url{http://cartilha.cert.br}}
do Centro de Estudos, Resposta e Tratamento de Incidentes de Segurança no Brasil (CERT.br)
%\nomenclature{CERT.br}{Centro de Estudos, Resposta e Tratamento de Incidentes de Segurança no Brasil}.
}


O comércio eletrônico possui outros suportes além do computador, tais como o celular e, mais recentemente, a TV digital. 
A compra de produtos pela TV não é algo novo. Atualmente existem
até canais específicos para tal atividade, no entanto, o usuário
precisa utilizar um outro canal para finalizar o processo de compra.
A TV Digital traz a facilidade de permitir que todo este processo
seja iniciado e finalizado diretamente do controle remoto da TV,
trazendo mais comodidade para os usuários, em uma modalidade
denominada \textit{T-Commerce}\nomenclature{T-Commerce}{Comércio pela TV Digital}.

As possibilidades dos recursos de interatividade da TV Digital (TVD) são 
\replaced{inúmeras}{praticamente infinitas}, dependendo da criatividade das produtoras de conteúdo
e desenvolvedores de \textit{software}. Um dos sonhos atualmente possíveis com as tecnologias
de TVD é a compra de produtos que estejam sendo exibidos em um programa de TV
convencional, como um tênis, uma bolsa, um quadro, ou qualquer outro.

Tendo em vista esta tendência crescente de provimento de serviços nas mais 
diversas plataformas e o sucesso de vários serviços de comércio eletrônico 
no Brasil e no mundo, a presente dissertação descreve
uma arquitetura para provimento de comércio eletrônico pela TV Digital (\textit{T-Commerce}).
A arquitetura proposta é composta por diversos serviços \textit{Web} que, juntos,
agregam todos os serviços que são disponibilizados aos usuários 
dos sistemas de \textit{T-Commerce} a serem desenvolvidos a partir de tal arquitetura.
Esta arquitetura é então enquadrada como uma Arquitetura Orientada a Serviços (\textit{Service Oriented Architecture} - SOA),
voltada para o provimento de serviços de \textit{T-Commerce}.

%http://www.mc.gov.br/noticias-do-site/23013-ministerio-das-comunicacoes-participa-de-seminario-sobre-apagao-analogico
Tal arquitetura foi elaborada devido não terem sido encontrados outros trabalhos
que tratem de uma proposta de comércio eletrônico para a TV Digital.
Assim, a proposta aqui apresentada serve como base para o desenvolvimento
de aplicações de \textit{T-Commerce}, visando a popularização de serviços
de comércio eletrônico pelo Sistema Brasileiro de TV Digital.
Considerando que o sinal analógico de TV Digital está previsto
para ser desligado em 2016, segundo previsão do Ministério das Comunicações\footnote{\url{http://goo.gl/nLVnW}},
todos os brasileiros que desejarem receber sinal de TV, precisarão de um
televisor com conversor integrado ou um conversor para conectar à um televisor convencional.
Desta forma, sistemas de \textit{T-Commerce} podem ter um grande alcance.

A arquitetura proposta, elaborada a partir de requisitos funcionais e não funcionais, 
serve de base para a implementação de aplicativos para comércio eletrônico,
para o qual são também elicitados requisitos funcionais e não funcionais.
Esses aplicativos são construídos com base em um modelo 
de \textit{templates} para definir a identidade
visual da mesma, permitindo que, ao ser alterado o \textit{template}, toda
a identidade visual da aplicação seja alterada.

A arquitetura e a aplicação de \textit{T-Commerce} propostas utilizam \textit{Web Services}
para disponibilizar funcionalidades aos usuários. Desta forma, 
tal proposta vai ao encontro da também crescente tendência de integração entre
\textit{Web} e TV. Tal integração é possível por meio de protocolos de comunicação
padronizados, como o caso dos protocolos HTTP e SOAP, que neste trabalho são 
objeto de implementações específicas.

Para esta integração entre \textit{Web} e TV, apresenta-se uma proposta
de \textit{framework} de comunicação de dados que possibilita a comunicação
entre aplicações de TV Digital e serviços disponíveis na \textit{Web},
o qual tem seu emprego demonstrado não somente por meio de aplicações de \textit{T-Commerce},
mas também de rastreamento de encomendas, leitor de RSS e outras.

\added{
Um \textit{framework} é um conjunto de classes que incorporam um projeto abstrato de soluções para 
uma família de problemas relacionados\cite{johnson1988designing}. O mesmo pode ser também denominado como
arcabouço, estrutura, esqueleto, suporte e outros termos.
}

Alguns dos requisitos do aplicativo de \textit{T-Commerce} supracitado são contemplados com a extensão
do \textit{framework} LuaOnTV para permitir o uso de temas e a adaptação
automática da interface de usuário da aplicação para diferentes 
tamanhos de TV ou até mesmo em dispositivos móveis como telefones celulares.


