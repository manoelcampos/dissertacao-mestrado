\subsection{Comparação entre as Propostas de Descoberta de Serviços}

Na Tabela \ref{tab:comparacao-descoberta-ws} é apresentada uma comparação entre as propostas de descoberta de serviços abordadas na Seção \ref{sec:propostas-descoberta-ws}. A primeira coluna apresenta as propostas avaliadas. A segunda indica se a arquitetura utilizada para a descoberta de serviços é distribuída ou não. A terceira indica se os repositórios de registro de serviço detectam e classificam os serviços inativos, para que somente os serviços ativos sejam disponibilizados aos usuários. A quarta indica como é feito o processo de registro/descoberta de serviços. A quinta informa se a proposta detecta, automaticamente, modificações nos serviços registrados. A sexta coluna indica se a proposta utiliza descrições semânticas para estender
o documento WSDL. A sétima indica se a proposta utiliza parâmetros de QoS para seleção de serviços.

\begin{center}
{\tiny
  \begin{tabular}{|p{2.5cm}|p{0.6cm}|p{2.5cm}|p{2.9cm}|p{3.5cm}|p{1.5cm}|p{0.4cm}|} %{|l|c|l|l|l|c|c|}
  \hline
	\textbf{Proposta} &\textbf{Distrib} &\textbf{Classifica Serviços Inativos}\par \textbf{Automaticamente} & 
	\textbf{Registro dos Serviços} &\textbf{Rastreia alterações nos WSs}\par \textbf{Automaticamente} &
	\textbf{Semântica} & \textbf{QoS}   \\

   \hline
	\textit{A Novel Interoperable Model}\par \textit{of Distributed UDDI} \cite{wu2008novel}\par Seção \ref{sec:novel-model-uddi} 
	& Sim 
	& Sim & Manual,\par realizado pelo provedor. 
	& Não.\par Somente os provedores podem informar sobre alterações. 
	& Não & Não\\

    \hline
	\textit{Active Web Service} \cite{treiber2007active}\par Seção \ref{sec:active-ws-reg} 
	& Sim 
	& Não.\par Somente os provedores podem informar sobre a desativação de um serviço, atualizando o \textit{feed} RSS/Atom.  
	& Manual.\par Os clientes devem assinar o \textit{feed} RSS/Atom do provedor, 
	para serem informados de novos serviços e atualizações,
	ou utilizar sites agregadores de \textit{feeds} de diversos provedores. 
	& Não.\par As alterações só são detectadas a partir do momento que o provedor 
	atualiza o \textit{feed} RSS/Atom de publicação dos serviços. 
	& Não & Não\\

    \hline
	 \textit{A New Description Model}\par \textit{of Web Service} \cite{wu2009new}\par Seção \ref{sec:ws-description-model}  
	& NI 
	& Não Informado 
	& Automático,\par mas não é informado como, apenas cita que com uso de OWL-S isto é possível 
	& Automático,\par mas não é informado como, apenas cita que com uso de OWL-S isto é possível 
	& Sim,\par usando OWL-S. & Sim\\

    \hline
	\textit{Finding Web Services} \cite{lausen2007finding}\par Seção \ref{sec:finding-ws} 
	& NI
	& Sim & Automatizado, por meio de um \textit{crawler} específico 
	& Não informado,\par mas o \textit{crawler} pode detectar tais alterações e atualizar as descrições do serviço.
	& Sim,\par obtendo informações implícitas (como a localização, obtida a partir do IP) e explícitas (informadas no WSDL) 
	& Sim \\

    \hline
	\textit{Toward Quality-Driven}\par \textit{Web Service Discovery} \cite{al2008toward}\par Seção \ref{sec:quality-driven-discovery} 
	& NI & Sim
	& Automatizado, por meio de Web \textit{crawling} usando um componente denominado \textit{Web Service Broker}, ou registro manual.
	& Não informado,\par mas o \textit{Web Service Broker} pode detectar tais alterações e atualizar as descrições do serviço.
	& Não & Sim \\

    \hline
   	 \textit{Semantic WS Offer Discovery}\par \textit{for E-Commerce} \cite{kopecky2008semantic}\par Seção \ref{sec:sws-offer} 
	 & NI & Não Informado & Manual.\par A descoberta é feita a partir dos serviços registrados, baseado em critérios do usuário.
	 & Não Informado & Sim,\par usando WSMO-\textit{Lite}, baseado no WSMO. & NI \\

    \hline
	\end{tabular}
	*NI = Não Informado

    \captionof{table}{Comparação das Propostas de Descoberta de Serviços}
    \label{tab:comparacao-descoberta-ws}
}
\end{center}
