\subsection{\textit{Reputation-Oriented Trustworthy Computing in E-Commerce} \cite{wang2008reputation}}

Em comércio eletrônico (\textit{E-Commerce}), a reputação de uma empresa na Internet é muito importante para garantir a confiança dos clientes
em realizar transações com as mesmas. O medo de ser enganado, ter dados confidenciais roubados, não receber o produto
comprado de acordo com as especificações anunciadas, excessiva demora na entrega
do produto e recebimento de spam, são fatores que podem fortemente influenciar o 
cliente a não comprar em uma loja virtual cuja reputação ele desconhece.

Segundo \cite{wang2008reputation}, existentes sistemas de \textit{E-Commerce} tem introduzido mecanismos de gerenciamento de confiança que proveem algumas informações de avaliação para os clientes. No entanto, mecanismos mais abrangentes devem ser providos para mais precisamente esboçar o nível de confiança dos vendedores em transações potenciais. A pesquisa desenvolvida em \cite{wang2008reputation} apresenta uma revisão sobre os mecanismos de avaliação de confiança baseados em reputação.

Recentemente, computação orientada a serviços (\textit{Service-Oriented Computing} (SOC)\nomenclature{SOC}{\textit{Service-Oriented Computing}}) tem emergido como uma importante tecnologia para disponibilizar diferentes serviços, de diferentes provedores. Consumidores podem procurar por serviços qualificados usando as capacidades de descoberta de serviço de um repositório, invocar um ou mais serviços e receber o resultado, tudo de forma integrada. Nesse ambiente, a reputação do provedor de serviços é uma grande peocupação do consumidor, antes de realizar uma transação. Estes fatores também são críticos para os repositórios, que devem manter recomendadas listas de confiáveis serviços e provedores.

Um sistema de \textit{E-Commerce} pode obter índices de confiança medindo a qualidade do serviço entregue, bem como obtendo avaliações a partir dos consumidores e de autoridades de gerenciamento de confiança.

\subsubsection{Categorias de Computação de Confiança}

De forma geral, existem duas classes de computação de confiança: computação de confiança orientada à segurança e computação de confiança não orientada à segurança. A última pode ser dividida em duas sub-classes: computação de confiança orientada socialmente e computação de confiança orientada a serviços.

Na computação de confiança orientada a segurança, a confiança provê um mecanismo para melhorar a segurança, cobrindo assuntos de autenticação, autorização, controle de acesso e privacidade. Segundo \cite{wang2008reputation}, confiança é o grau pelo qual um objeto alvo (tal como um software, um dispositivo, um servidor, ou qualquer dado que eles entreguem) é considerado seguro.

Tanto na computação de confiança orientada socialmente e orientada a serviços, pode-se definir confiança em termos de crença de confiança e comportamento de confiança \cite{wang2008reputation}. Crença de confiança entre duas partes é a medida em que uma parte acredita que a outra é confiável, em uma dada situação. "Confiável" significa que uma parte deseja e é capaz de agir no interesse do outro. Confiança entre duas partes é a medida em que uma parte depende da outra em uma dada situação com um sentimento de garantia relativa, até mesmo pensando que consequências negativas são possíveis. Se uma crença de confiança significa que "A acredita que B é confiável", isto irá guiar para um comportamento de confiança, tal como "A confia em B". \cite{wang2008reputation}

No contexto de comércio e serviços eletrônicos, avaliação de confiança usualmente ocorre via avaliação de reputação, baseada no serviço, transação ou histórico de interações. No contexto de ambientes socialmente orientados, estudos têm focado em esboçar e dereviar o relacionamento entre as partes e conduzir a avaliação  de confiança final. Já a confiança orientada a serviços é um mecanismo para arquivar, manter e raciocinar sobre a qualidade do serviço (QoS) e das interações. 

Avaliação de confiança baseada em reputação tanto para computação de confiança orientada socialmente e orientada a serviços.  Em geral, um serviço ganha boa reputação após um longo perído de bons índices de qualidade (QoS).

\subsubsection{Sistemas de Avaliação de Reputação Mútua}

Um sistema bastante conhecido, utilizado por lojas virtuais como eBay\footnote{\url{http://www.ebay.com}} e Mercado Livre\footnote{\url{http://www.mercadolivre.com.br}}, que fornecem serviços do tipo \textit{consumer-to-consumer}\footnote{As lojas citadas, intermediam negociações entre vendedores e compradores.}, possuem um sistema de avaliação de reputação, como abordado em \cite{wang2008reputation}, onde o comprador, após finalizar a negociação (tendo recebido o produto ou não), pode avaliar a qualidade do serviço, prestado pelo vendedor, como: positiva, negativa ou neutra. O vendedor também pode fazer a mesma avaliação do comprador. Um resumo das avaliações dos usuários são disponibilizadas para todos os que têm acesso ao sistema, que podem decidir se querem fazer negócio com uum determinado usuário ou não. O sistema permite ainda a realização de réplicas para as avaliações e comentários. Em casos de qualificações negativas, a empresa que disponibiliza o ambiente (no caso do exemplo, eBay ou Mercado Livre), que atua como autoridade de gerenciamento de confiança, pode intervir e aplicar sanções contra a parte que possa ter infringido algum termo de uso do serviço, como por exemplo, expulsar o usuário da comunidade.


