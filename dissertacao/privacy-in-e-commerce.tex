\subsection{\textit{Privacy in E-Commerce} \cite{berendt2005privacy}}

Privacidade é um dos requisitos básicos para sistemas de comércio eletrônico (\textit{E-Commerce}). Muitos dos recursos disponíveis para garantir tal privacidade foram relatados na Seção \ref{e-commerce-secutiry}. No entanto, por mais segurança que os sistema provejam, os usuários tem grande responsabilidade na segurança das suas informações. Eles precisam manter seus sistemas operacionais, navegadores e anti-vírus atualizados, para garantir que os mesmos contam com as últimas atualizações e correções de falhas de segurança, além de tomar cuidados durante o acesso a sistemas que solicitam informações pessoais e sigilosas como número de cartão de crédito, senhas, endereço, telefone e outros dados. Os sistemas de comércio eletrônico, em especial, sempre devem utilizar tecnologias de assinatura digital, como apresentado na Seção \ref{e-commerce-secutiry}, para provar sua identidade para seus clientes. No entanto, técnicas maliciosas como o \textit{Phishing} podem ser utilizadas para roubar a identidade de empresas na Internet, permitindo que um \textit{site} falso se passe pelo \textit{site} de uma determinada empresa \footnote{Mais informações sobre Phishing e segurança na Internet podem ser encontradas na cartilha disponível em \url{http://cartilha.cert.br}, desenvolvida pelo Centro de Estudos, Resposta e Tratamento de Incidentes de Segurança no Brasil (CERT)\nomenclature{CERT}{Centro de Estudos, Resposta e Tratamento de Incidentes de Segurança no Brasil}.}.

Considerando que o comportamento do usuário é um fator importantíssimo para a segurança de suas informações, a pesquisa desenvolvida em \cite{berendt2005privacy} apresenta resultados sobre o comportamento de usuários ao fornecer informações pessoais, em sistemas de \textit{E-Commerce}.
A pesquisa relata que, dadas certas circustâncias, usuários esquecem facilmente sobre suas preocupações de privacidade e informam até mesmo os detalhes mais pessoais sem uma razão convincente para fazer isto, ainda mais quando o serviço acessado é de entretenimento ou são oferecidos benefíciso em troca de informações. Por exemplo, o usuário pode ser sugerido a fornecer dados pessoais relevantes para receber descontos ou recomendações de produtos.

A pesquisa relata que as preferência de privacidade declaradas pelos usuários, não tem impacto no comportamento da maioria deles. Isto porque, enquanto os usuários estão interagindo em uma transação \textit{on-line}, eles frequentemente não monitoram ou controlam suas ações suficientemente \cite{berendt2005privacy}, não estando alerta sobre os comportamentos que eles consideram inseguros.
