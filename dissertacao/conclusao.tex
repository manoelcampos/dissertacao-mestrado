\chapter{Conclusões e Trabalhos Futuros} \label{cap:conclusao}

\section{Conclusões}

\added{As implementações apresentadas nesta dissertação são todas originais, por
não haver nenhuma outra implementação (pelo menos de código aberto)
para o sub-sistema Ginga-NCL dos protocolos HTTP e SOAP, de
um \textit{framework} de componentes gráficos e de aplicações de
\textit{T-Commerce}. Tais implementações permitem o desenvolvimento
de aplicações dos mais variados tipos para o Ginga-NCL, alavancando
o desenvolvimento de aplicações interativas para o SBTVD (ISDB-TB).}

Os módulos NCLua HTTP e NCLua SOAP facilitam a convergência entre \textit{Web} e TV, escondendo
os detalhes de implementação dos protocolos HTTP e SOAP do desenvolvedor
de aplicações de TVDi, permitindo o surgimento de novas aplicações interativas 
que fazem uso de conteúdo da \textit{Internet}.
O NCLua SOAP permitiu o desenvolvimento de aplicações de \textit{T-Government}
como apresentado em \cite{tgov2010barbosa}, sistema de recomendação\cite{gatto2010BIPODiTVR} entre outras. 
Na página do projeto, em http://ncluasoap.manoelcampos.com,
existe um \textit{link} para o fórum de discussão do módulo, onde alguns usuários relatam a
utilização do mesmo, por exemplo, em aplicações de \textit{T-Learning}.

Atualmente, o processo de obtenção dos dados para realizar a chamada a um método remoto em um \textit{Web Service}
ainda é praticamente todo manual, no entanto, 
após o desenvolvedor obter tais dados para o método remoto desejado, a realização da requisição é bastante simplificada,
principalmente pelo fato de Lua ser uma linguagem de tipagem dinâmica \cite{ierusalimschy2006programming}, 
não obrigando a declaração de variáveis com seus respectivos tipos.
O \textit{feedback} dos usuários tem mostrado que o módulo está sendo bastante útil para 
a comunidade, além de permitir a evolução do mesmo.


%\textbf{APRESENTAR CONCLUSÕES A AVALIAÇÕES DE RESULTADOS DA ARQUITETURA E DA APP DE T-COMMERCE}

A arquitetura proposta serve de base para o desenvolvimento de serviços de \textit{T-Commerce} para o SBTVD,
mostrando como diversos serviços podem ser integrados em uma única arquitetura para um propósito final.
Tal arquitetura também serve como um estudo de caso dos percalços enfrentados para a elaboração
de um ambiente de desenvolvimento e testes de aplicações para o SBTVD, mostrando
as ferramentas necessárias para isto.

As análises de desempenho apresentadas dos protocolos implementados mostraram como a solução proposta
é eficiente em uso de processador e memória RAM, sendo uma solução ideal para uso em 
ambientes restritos de recursos, como os \textit{Set-top Boxes}.

Os protocolos implementados são a base da integração entre \textit{Web} e TV
e estão sendo bastante úteis em diversos outros trabalhos, como apresentado
no Capítulo \ref{cap:ncluasoap}, mostrando como os resultados alcançados se tornaram
importantes para a comunidade de TV Digital no Brasil e na América Latina, 
onde quase todos os países já adotaram o ISDB-TB.

A aplicação de \textit{T-Commerce} desenvolvida delineia o processo de desenvolvimento
de aplicações que fogem do trivial, mostrando como aplicar recursos como orientação
a objetos na linguagem Lua, \added{uso do canal de retorno/interatividade no SBTVD, 
uso de arquivos XML e arquivos de dados em formato Lua, além de outros recursos}. 
O uso do \textit{framework} LuaOnTV para construção
de interfaces gráficas de usuário é bastante flexível e dá suporte a múltiplos
dispositivos, adaptação automática das dimensões dos componentes gráficos
da aplicação, além do uso de \textit{templates} para permitir uma 
definição centralizada da identidade visual da aplicação.
Tal recurso de \textit{templates} permite alterar a identidade visual da aplicação facilmente, 
podendo-se ter \textit{templates} diferentes para tipos de dispositivos diferentes (como TV's e celulares).

Outras aplicações foram desenvolvidas, como o NCLua RSS \textit{Reader} e o Rastreador de Encomendas
que servem como prova de conceito dos protocolos como HTTP e SOAP, implementados nesta dissertação.

Os objetivos apresentados na Seção \ref{sec:objetivos} foram todos alcançados,
apresentando uma solução completa, desde o ambiente de desenvolvimento, até 
a realização de análises de desempenho das aplicações desenvolvidas.

\added{
A Tabela \ref{tab:obj-espec-alcancados} apresenta uma descrição de como tais objetivos foram alcançados.
\begin{center}
	\begin{tabular}{|p{2cm}|p{13cm}|}
	  \hline 
		\textbf{Objetivo Específico} & \textbf{Como foi alcançado} \\
		\hline 
		A & Os requisitos funcionais e não funcionais foram elicitados, tendo
		servido de guia para o desenvolvimento da arquitetura proposta; \\
		\hline 
		B & a arquitetura foi proposta e desenvolvida, apresentando-se
		a mesma por meio de figuras e diagramas UML, além da montagem de uma
		distribuição GNU/Linux contendo todo o ambiente de desenvolvimento elaborado;  \\	
		\hline 
		C & um \textit{framework} de comunicação de dados foi desenvolvido, composto
		pelos módulos NCLua HTTP e NCLua SOAP, que implementam os protocolos HTTP e SOAP, respectivamente,
		sendo a base de toda a interoperação das aplicações desenvolvidas com diferentes provedores de serviços \textit{Web}; \\
		\hline 
		D & as diferentes aplicações de TVDi propostas foram desenvolvidas e apresentadas, servindo como prova
		de conceito do uso dos \textit{frameworks} desenvolvidos ou estendidos; \\
		\hline 
		E & o \textit{framework} LuaOnTV foi estendido, incluindo-se suporte a temas. O recurso
		de temas foi utilizado na aplicação de \textit{T-Commerce} desenvolvida. 
		A adaptação automática da interface da aplicação foi testada em ambiente
		virtual de TV Digital (o Ginga \textit{Virtual Set-top Box}) 
		com diferentes resoluções de tela. Em tal ambiente, a adaptação automática do tamanho
		da interface gráfica da aplicação ao tamanho da tela do televisor
		mostrou-se satisfatória, possibilitando que a aplicação de adapte automaticamente
		a diferentes dispositivos receptores de TVD; \\
		\hline 
		F & um modelo de desenvolvimento de aplicações de TVDi baseado em \textit{templates}
		foi elaborado e apresentado. Tal modelo foi utilizado na aplicação de \textit{T-Commerce} desenvolvida,
		o que permitiu uma agilidade no desenvolvimento da mesma, definindo
		um conjunto de componentes gráficos básicos e uma identidade visual
		comum a todas as telas da aplicação desenvolvida.\\								
		\hline 
	\end{tabular}
	\captionof{table}{Objetivos Específicos Alcançados}
	\label{tab:obj-espec-alcancados}
\end{center}
}

\section{Trabalhos Futuros}

Como trabalhos futuros, pretende-se concluir a implementação do \textit{parse} automático do 
documento WSDL e geração de \textit{stubs} Lua, contendo funções \textit{proxies} para realizar a chamada aos métodos remotos
(semelhante a ferramentas como o \textit{wsdl2java}\footnote{http://ws.apache.org/axis/java/user-guide.html}, 
pretende-se transformar o \textit{script} \textit{wsdlparser} em um \textit{wsdl2nclua})
e incluir tratamento de exceções para permitir que as aplicações
de TVDi possam emitir mensagens amigáveis ao usuário quando uma requisição HTTP falhar.

O módulo NCLua HTTP permite que seja utilizado qualquer método HTTP em uma requisição, no entanto,
apenas os método \textit{GET} e \textit{POST} foram testados. Desta forma, pretende-se realizar testes
de conformidade utilizando-se os métodos HTTP \textit{OPTIONS, HEAD, PUT} e \textit{DELETE}. Pretende-se também
implementar mais funcionalidades no módulo, como realizar redirecionamentos automaticamente
a partir de respostas HTTP como 301 e 302, além de implementar alguns recursos do HTTP/1.1, como
conexões persistentes.



A arquitetura também pode ser estendida nos pontos a seguir:

\begin{itemize}
	\item incluir suporte a metadados na aplicação para que a
mesma possa oferecer produtos vinculados
a um programa televisivo, permitindo que
a oferta do produto seja mostrada 
ao telespectador em momento determinado no
arquivo de metadados, utilizando os recursos
de sincronização de mídias existente na linguagem NCL;
   \item estudar a viabilidade e uso do protocolo de autorização \textit{Open Authentication} (oAuth)\footnote{\url{http://oauth.net}},
que é bastante utilizado atualmente em redes sociais,
permitindo ao usuário ter um login e senha únicos para acesso a diferentes serviços,
agilizando o processo de login (caso o usuário decida salvar localmente, de forma criptografada, 
as informações de autorização);
   \item estender a arquitetura para um modelo baseado em descoberta de serviços que
possibilite a integração de diferentes lojas virtuais
que implementem a arquitetura aqui proposta,
utilizando recursos como a linguagem BPEL (\textit{Business Process Execution Language})\nomenclature{BPEL}{\textit{Business Process Execution Language}}
para permitir a composição de serviços e tornar tal integração transparente para aplicação de TVDi;
  \item utilizar a extensão \textit{WS-Security}\cite{oasis-wssecurity} para permitir a segurança na comunicação entre os \textit{Web Services} e as aplicações cliente;
  \item criar ferramenta, para execução do lado da emissora de TV, que permita o envio dos produtos em oferta
  via \textit{broadcast}, possibilitando a usuários, sem conexão com a \textit{Internet}, visualizarem tais produtos
  na tela de seu equipamento de TVD.
\end{itemize}

