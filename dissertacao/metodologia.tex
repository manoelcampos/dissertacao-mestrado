\section{Metodologia} \label{sec:metodologia}

Para o desenvolvimento da presente dissertação foi feito um levantamento bibliográfico
sobre as tecnologias, linguagens, ferramentas e trabalhos relacionados para nortear
a implementação da arquitetura e solução proposta.

O projeto, análise e desenvolvimento da solução seguiu o
processo de desenvolvimento em cascata juntamente com o processo de componentização\cite{sommerville2011soft}.
O processo em cascata foi adotado pois os requisitos estavam bem definidos, havendo \replaced{pequena}{pouca} probabilidade de mudanças radicais.
Já o processo de componentização foi também adotado devido ao uso de alguns componentes reutilizáveis
(como \textit{Web Services} próprios e de terceiros) realizando a integração destes diversos
componentes.

No processo de desenvolvimento em cascata, foram seguidas as seguintes etapas:
\begin{itemize}
	\item especificação de requisitos - foram definidos os requisitos funcionais e não funcionais, que são apresentados ao longo do trabalho;
	\item projeto - foi adotado o paradigma de Orientação a Objetos (OO)\nomenclature{OO}{Orientação a Objetos}, utilizando-se a \textit{Unified Modeling Language} (UML)\nomenclature{UML}{\textit{Unified Modeling Language}} para modelagem, com o auxílio de ferramenta CASE (\textit{Computer-Aided Software Engineering})\nomenclature{CASE}{\textit{Computer-Aided Software Engineering}};
	\item implementação - a implementação seguiu o paradigma OO, conforme definido na fase de projeto;
	\item testes - foram feitos testes de interoperabilidade/integração para verificar se os componentes
	da arquitetura estavam se comunicando conforme o esperado.
\end{itemize}


Optou-se, para o desenvolvimento do projeto, pelo paradigma de orientação a objetos pois o mesmo 
permite um alto grau de reutilização de código, tornando o mesmo mais organizado
e fácil de manter.

Para a comunicação entre os componentes da arquitetura da solução, foi escolhido o protocolo de comunicação SOAP, uma vez que a aplicação de TV Digital desenvolvida
faz parte de uma arquitetura distribuída e precisa realizar comunicação 
com servidores \textit{Web} por meio da \textit{Internet}. Assim, foi preciso usar um protocolo
que não tivesse problemas de bloqueio em \textit{firewalls}. Desta forma, o protocolo SOAP foi escolhido
também por ser padrão do \textit{World Wide Web Consortium} (W3C)\nomenclature{W3C}{\textit{World Wide Web Consortium}}.

Para as aplicações de TV Digital, foi escolhido o sub-sistema Ginga-NCL do \textit{middleware} Ginga
do Sistema Brasileiro de TV Digital (SBTVD)\nomenclature{SBTVD}{Sistema Brasileiro de TV Digital},
devido à grande quantidade de documentos, fóruns de discussão, ferramentas e exemplos disponíveis
em relação à outra vertente de desenvolvimento utilizando a linguagem Java no sub-sistema Ginga-J.
Considerando-se ainda que o Ginga-NCL é o único sub-sistema obrigatório para dispositivos móveis,
isto foi decisivo para a escolha, pois assim, a arquitetura e as aplicações desenvolvidas
podem, em tese, ser executadas em receptores (conversores/\textit{Set-bop Boxes}) de TV Digital fixos, móveis ou portáteis.

Com a escolha do Ginga-NCL, foi necessária a implementação do protocolo SOAP para este sub-sistema,
utilizando-se a linguagem Lua, uma vez que o primeiro só disponibiliza protocolos até a camada de transporte
do modelo OSI/ISO, como o TCP. Com isto, foi necessário realizar testes de interoperabilidade
entre aplicações de TV Digital desenvolvidas com as linguagens NCL e Lua e \textit{Web Services}
desenvolvidos em diferentes plataformas e linguagens, pois um dos grandes benefícios
do protocolo SOAP é a integração de sistemas heterogêneos. Assim, tais testes foram
necessários para garantir esta interoperabilidade.

Os requisitos da aplicação foram levantados tomando por base as funcionalidades
existentes na grande maioria dos \textit{Web sites} de comércio eletrônico
disponíveis na \textit{Internet} e bastante difundidos no Brasil.

Para a construção da interface de usuário das aplicações de TV Digital, optou-se pela utilização
do \textit{framework} LuaOnTV para abstrair as primitivas gráficas para renderização da interface.
A mesma foi projetada baseada em recursos de \textit{templates}, o que permite a personalização
e adaptação automática para diferentes resoluções de tela, tendo sido estendido
o LuaOnTV para incluir tais recursos.


