\section{Sistema Brasileiro de TV Digital (SBTVD)}

O Sistema Brasileiro de TV Digital (SBTVD) atualmente é o mais
avançado do mundo. A necessidade de o Brasil implantar um sistema de TV Digital
levou o Governo Federal a emitir o Decreto 4.901, de 26 de novembro de 2003, instituindo
o referido sistema.

A partir daí, o governo passou a investir diretamente na pesquisa de tecnologias
para TV Digital. Considerando que já \replaced{havia}{haviam} outros padrões de TV Digital
no mundo, como o americano ATSC (\textit{Advanced Television Systems Committee})\nomenclature{ATSC}{\textit{Advanced Television Systems Committee}}, 
o europeu DVB (\textit{Digital Video Broadcast})\nomenclature{DVB}{\textit{Digital Video Broadcast}} 
e o japonês ISDB-T (\textit{Integrated Services Digital Broadcasting Terrestrial})\nomenclature{ISDB-T}{\textit{Integrated Services Digital Broadcasting Terrestrial}}, tais estudos concluíram que o melhor para o país seria a adoção do sistema japonês ISDB-T,
que seria estendido para atender às características e necessidades do Brasil.

Desta forma, o Decreto 5.820, de 29 de junho de 2006 estabelece o uso do ISDB-T
como base para o Sistema Brasileiro de TV Digital Terrestre (SBTVD-T)\nomenclature{SBTVD-T}{Sistema Brasileiro de Televisão Digital Terrestre}.
O decreto ainda define a criação do Fórum SBTVD (Fórum do Sistema Brasileiro de TV Digital)\footnotetext{\url{http://www.forumsbtvd.org.br}}.
Tal fórum, como menciona o decreto supracitado, tem como atribuições: 

\begin{quote}
"a assessoria acerca de políticas e assuntos técnicos referentes à aprovação de inovações
tecnológicas, especificações, desenvolvimento e implantação do SBTVD-T".
\end{quote}
 
O mesmo foi composto, como cita o decreto: 
\begin{quote}
"entre outros, por representantes do setor de
radiodifusão, do setor industrial e da comunidade científica e tecnológica"
\end{quote}

Dentre as contribuições do Brasil para a área de TVD, destaca-se a construção do Ginga\footnote{\url{http://www.ginga.org.br}}
como padrão de \textit{middleware} para aplicações interativas e não interativas.
O Ginga é uma inovação nacional, atuando como responsável
pela execução e controle do ciclo de vida das aplicações interativas, abstraindo o sistema operacional
e o hardware para elas, além de realizar tarefas especializadas como a disponibilização de \textit{Application Programming Interfaces} (API's) para
facilitar o desenvolvimento de aplicações interativas\cite{soares2009programando}.

O Ginga é composto por dois sub-sistemas: um para a execução de aplicações declarativas, denominado
Ginga-NCL\footnote{\url{http://www.gingancl.org.br}}, e outro para execução de aplicações procedurais, 
denominado Ginga-J\footnote{\url{http://dev.openginga.org}}.
Aplicações declarativas são aquelas implementadas utilizando-se uma linguagem declarativa, como
XML, de forma não algorítmica, apenas declarando elementos que comporão a mesma.
Tais linguagens abstraem muitos detalhes do desenvolvedor. 
Aplicações procedurais são aquelas implementadas utilizando uma linguagem procedural, como Java,
onde o desenvolvedor precisa escrever um algoritmo e especificar cada operação a ser realizada,
necessitando de conhecimento em linguagens de programação.

O sub-sistema Ginga-NCL permite a construção de aplicações declarativas por meio da linguagem NCL,
a \textit{Nested Context Language}\nomenclature{NCL}{\textit{Nested Context Language}}\footnote{\url{http://www.ncl.org.br}}. 
Esta é conhecida como uma linguagem de cola, que permite criar aplicações declarativas
juntando-se diferentes tipos de mídias como imagens, texto, hipertexto (HTML), vídeos
e outros. Um de seus principais recursos é a sincronização de mídias, muito
importante no contexto de aplicações interativas. Tal recurso garante que, por exemplo,
uma legenda seja sincronizada com um vídeo em exibição, ou que uma imagem
apareça depois de determinado tempo que o vídeo iniciou.
As aplicações NCL podem ter seu poder estendido com a inclusão de mídias especiais:
os \textit{scripts} Lua\footnote{\url{http://www.lua.org}}, conhecidos em aplicações NCL como NCLua. Tais \textit{scripts}
adicionam características procedurais às aplicações NCL.

O outro sub-sistema que compõe o Ginga é o chamado Ginga-J, que permite a construção
de aplicações interativas utilizando a linguagem Java. Os outros padrões de \textit{middleware}
de TV Digital pelo mundo, na parte procedural, normalmente utilizam a linguagem Java e a API JavaTV.
No entanto, os fabricantes, para embarcarem a máquina virtual Java
e tal API nos conversores de TV Digital precisam pagar \textit{royalities} 
à Oracle, empresa que atualmente detém os direitos sobre a marca Java.
Com isto, o preço dos conversores é encarecido. Além disto, 
tal API é baseada no \textit{toolkit} gráfico AWT, que é bastante
antigo e não possui componentes com visual bonito e elegante.
Devido às questões de pagamento de \textit{royalities}, o Fórum SBTVD
decidiu fazer um acordo entre a então Sun, hoje Oracle, para 
a definição de uma nova API para ser utilizada pelo SBTVD, a qual
foi batizada de JavaDTV\footnote{\url{http://www.forumsbtvd.org.br/materias.asp?id=75}}. Tal API é livre de \textit{royalities} 
e é baseada no \textit{toolkit} gráfico LWUIT\footnote{\url{http://lwuit.java.net}}, o \textit{Light Weight UI Toolkit}\nomenclature{LWUIT}{\textit{Light Weight UI Toolkit}}.
O LWUIT é um \textit{toolkit} que possui componentes bonitos e elegantes
e é o mesmo utilizado em aplicações para celulares.

Com as contribuições brasileiras, o ISDB-T e o SBTVD-T passaram a compartilhar uma denominação internacionalmente
conhecida como ISDB-TB, onde "B" representa as contribuições do Brasil para o ISDB-T\cite{isdb-tb}.

As melhorias resultantes da parceria realizada levaram o ISDB-TB a se tornar norma internacional de TV Digital no ITU-T (\textit{International Telecommunications Union - Telecommunication Standardization Sector})\cite{info-gingancl-h761}\cite{idgnow-ginga-padrao-itu}.
O Ginga se tornou padrão de \textit{middleware} para TVD, relativo à recomendação ITU-T J.200\cite{ginga-itu-j200}.
Seu sub-sistema declarativo Ginga-NCL se tornou recomendação ITU-T J.201\cite{gingancl-itu-j201}
e seu sub-sistema imperativo Ginga-J se tornou recomendação ITU-T J.202\cite{gingaj-itu-j202}.

Ressalta-se, também, a recente normatização, no âmbito do ITU-T, referente a IPTV, conhecida como H.761, baseada fortemente no \textit{middleware} Ginga
e seu sub-sistema Ginga-NCL. Desta forma, o Ginga pode chegar a novas plataformas que não apenas a TV aberta, e as aplicações interativas 
e não interativas desenvolvidas em NCL/Lua para o SBTVD podem também ser executados em sistemas de IPTV que adotem o Ginga-NCL.
