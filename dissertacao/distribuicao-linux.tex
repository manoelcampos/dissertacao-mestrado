\section[Distribuição GNU/Linux p/ desenvolvimento de aplicações]{Distribuição GNU/Linux para desenvolvimento, execução e teste de aplicações de TV Digital}

Com base no trabalho apresentado em \cite{soset}, foi gerada uma distribuição GNU/Linux
contendo todo o ambiente de desenvolvimento necessário para a construção das
aplicações apresentadas ao longo deste trabalho.
O ambiente contém todas as ferramentas apresentadas em \ref{sec:dev-env}. 
Tal distribuição GNU/Linux foi
elaborada com o intuito de facilitar a montagem do ambiente de desenvolvimento
necessário para a construção de aplicações NCL/Lua para a TV Digital.

A comunidade Ginga no Portal do Software Público disponibiliza
uma implementação de referência do sub-sistema Ginga-NCL em forma
de uma máquina virtual já com tal implementação compilada e instalada. 
Tal máquina virtual facilita bastante a montagem do ambiente
de desenvolvimento, uma vez que o processo de compilação
de tal implementação é bastante extenso, dependendo de diversos
softwares e bibliotecas, não sendo um processo trivial de 
ser executado, principalmente para usuários menos
experientes com distribuições GNU/Linux e ferramentas
de compilação de linha de comando. No entanto,
o uso de uma máquina virtual adiciona um \textit{overhead} 
de tempo no processo de teste das aplicações, uma vez
que os arquivos das mesmas precisam ser enviados via SSH para a máquina
virtual, apesar de tal processo ser automatizado com o \textit{plugin} NCL Eclipse.

Desta forma, a instalação da implementação de referência do Ginga-NCL diretamente no 
sistema operacional utilizado pelo desenvolvedor para as suas tarefas rotineiras
(como envio de \textit{e-mail's}, elaboração de documentos em \textit{suites} de escritórios, etc) 
e de desenvolvimento de sistemas agiliza bastante o processo de execução e teste
das aplicações interativas, pois, como o Ginga é instalado localmente na máquina real,
não há processo de transferência de arquivos para poder executar as aplicações.
Com isto, a execução das aplicações é praticamente instantânea, além de
obter-se melhor desempenho executando o Ginga nativamente no sistema 
operacional da máquina real, uma vez que o uso de uma máquina
virtual obviamente requer o consumo de mais memória RAM e processador
que executar o Ginga nativamente em uma máquina real.

A distribuição desenvolvida foi baseada na versão 10.10 do Ubuntu e permite
o uso do ambiente de desenvolvimento sem a necessidade de instalação do mesmo
na máquina do desenvolvedor, podendo ser dado \textit{boot} na máquina por meio
de um CD contendo tal distribuição, conhecido como \textit{Live CD}. Ela pode
ser instalada em uma máquina real, onde o usuário poderá utilizar tal
distribuição como seu sistema operacional \textit{Desktop} para
a realização de suas tarefas rotineiras e de desenvolvimento e ainda
pode ser instalada em uma máquina virtual, já com o ambiente gráfico e todas
as ferramentas de desenvolvimento necessárias.

A versão da implementação de referência do Ginga-NCL embarcada na distribuição
é a última (até a data de entrega de tal dissertação), a 0.11.2 revisão 23, disponível
na Comunidade Ginga do Portal do Software Público\footnote{\url{http://svn.softwarepublico.gov.br/trac/ginga/wiki/Building\_Wiki\_GingaNCL}}.

O processo de criação da distribuição GNU/Linux consistiu basicamente em:
\begin{itemize}
	\item instalar distribuição Ubuntu 10.10 em uma máquina virtual utilizando a ferramenta Virtual Box 4.0;
	\item baixar e compilar a implementação de referência do Ginga-NCL em tal máquina virtual, juntamente
	com todas suas dependêncais;
	\item instalar ferramentas de desenvolvimento e suas dependências (como JDK e JRE);
	\item configurar ferramentas de desenvolvimento para permitir a execução local de aplicações NCL/Lua;
	\item gerar uma nova distribuição a partir do ambiente criado, já com todas as ferramentas instaladas,
	utilizando o \textit{software} \textit{RemasterSys}\footnote{\url{http://remastersys.sourceforge.net}}.
\end{itemize}

Tal distribuição pode ser \replaced{obtida}{baixada} pelo \textit{site} do Laboratório de TV Digital Interativa da Universidade de Brasília\footnote{\url{http://labtvdi.unb.br}}.
