\subsection{\textit{Finding Web Services} \cite{lausen2007finding}} \label{sec:finding-ws}

Segundo \cite{lausen2007finding}, um documento WSDL provê meios para descrever as propriedades da interface técnica como um documento XML, todavia, não provê suporte para todas as tarefas durante o ciclo de vida de um \textit{Web Service} como: publicação, descoberta, determinação dos detalhes da sequência de invocação, segurança, monitoramente, etc. Atualmente estes detalhes são tratados por desenvolvedores e engenheiros de software.

Quanto à descoberta de serviços, existem várias propostas na literatura de como estender o WSDL para adicionar detalhes semânticos, como a proposta em \cite{wu2009new}, apresentada na Seção \ref{sec:ws-description-model}.
Em \cite{lausen2007finding} é apresentada uma proposta de processo completamente automatizado para descoberta de serviços, onde as informações disponíveis para o serviço são obtidas, integradas em um modelo para permitir a descoberta em um conjunto atualizado de serviços. Neste modelo, descrições do serviço são criadas, principalmente, por processos de análise automáticos e pelo \textit{feedback} da comunidade, no lugar de usar conhecimento de engenheiros de software.

Segundo \cite{lausen2007finding}, baseado na origem de descoberta de serviços e em trabalhos anteriores, as três maiores propostas para descobrir \textit{Web Services} públicos são apresentadas a seguir.

\begin{itemize}
	\item UDDI como repositórios centralizados padrões, que possuem uma base de registro de \textit{Web Services};
  \item Diretórios de serviços que descobrem serviços usando \textit{crawlers}\nomenclature{Crawler}{Programa de computador, também conhecido como spider ou bot, utilizado para varrer a Web e permitir a catalogação das páginas existentes, cujo resultado é utilizado por motores de busca como \textit{Google} e \textit{Yahoo}.} específicos ou registro manual de serviços, oferecendo uma interface Web para a realização de buscas;
  \item Motores de busca padrões que são capazes de restringir a busca de forma a obter descrições WSDL. Embora isto não garante que serão encontrados serviços, isto possivelmente provê a maior cobertura.
\end{itemize}

A escolha de uma das propostas pode ser feita baseada, segundo \cite{lausen2007finding}, em dois aspectos: o número de serviços a encontrar e a qualidade e quantidade de informações associadas a eles, como explicado a seguir.

\begin{itemize}
	\item Número de Serviços: no caso do UDDI, os repositórios públicos começaram a ser desligados no início de 2006, logo, não foi possível realizar uma análise do serviço, no entanto, trabalhos anteriores reportaram que apenas um terço dos 1200 serviços registrados continham referências para arquivos WSDL válidos. 
A Tabela \ref{tab:ws-find-portals} mostra o número de \textit{Web Services} encontrados em diferentes portais Web e motores de busca. A segunda coluna
indica se o portal informa sobre a situação do serviço (se está ativo ou não),
a terceira mostra se o portal categoriza os serviços registrados,
a quarta indica o total de \textit{Web Services} registrados, e a última mostra o total de serviços ativos.
Para determinar o número de serviços disponíveis, via motores de busca como \textit{Google} e \textit{Yahoo}, foram detectados dois problemas: a inexistência de meios de obter todos os resultados\footnote{Segundo \cite{lausen2007finding}, \textit{Google} e \textit{Yahoo} mostram apenas os primeiros 1000 resultados por busca.} e de formular uma pesquisa para obter apenas descrições de \textit{Web Services} (WSDL's).
 \item Informações Disponíveis: devido a falta de acesso aos repositórios UDDI públicos, como explicado na Seção \ref{sec:uddi}, não é possível avaliar a qualidade das informações. Todavia, em \cite{kim2004survey} é relatada baixa qualidade de tais informações. Motores de busca como o \textit{Google} não coletam informações específicas relacionadas aos serviços. Os diretórios de \textit{Web Services}, como o \textit{XMethods}\footnote{\url{http://www.xmethods.com}}, incluem mais informações. Estes incluem dados sobre preço, sistema de pontuação e avaliação, informações textuais e \textit{links} para documentação online.
\end{itemize}

\begin{table}[ht!]
  \begin{center}
  \setlength{\belowcaptionskip}{10pt} % espao entre caption e tabela
  \footnotesize {
    \begin{tabular}{|p{3cm}|p{2.5cm}|p{2cm}|p{1.8cm}|p{1.5cm}|}
	  \hline
	  \textbf{Repositório} & \textbf{Inf. Atividade} & \textbf{Categorizado} & \textbf{Nº WSDLs} & \textbf{Acessíveis} \\
	  \hline
	  \textit{RemoteMethods} & N & S & 319 & 205\\
	  \hline
	  \textit{StrikeIron} & S & S & 638 & 508\\
	  \hline
	  \textit{Woogle} & S & S & 751 & 312\\
	  \hline
	  \textit{XMethods} & N & N & 505 & 460\\
	  \hline
	  \textit{programableWeb} & S & N & 80 & 77\\
	  \hline
	  \textit{Google} & N & N & 26200 & N/A\\
	  \hline
	  \textit{Yahoo} & N & N & 61800 & N/A\\
	  \hline
	  \textit{Alexa} & N & N & 30846 & 3630\\
	  \hline
    \end{tabular}
  }
  \caption{Número de \textit{WebServices} em Portais de Busca (adaptada de \cite{lausen2007finding}).}
  \label{tab:ws-find-portals}
  \end{center}
\end{table}

Com a descontinuidade dos repositórios UDDI públicos, as opções para localização de \textit{Web Services} são portais específicos e motores de busca, como apresentado na Tabela \ref{tab:ws-find-portals}. No entanto, são poucas as opções e muitos dos portais não incluem muitas informações a respeito dos serviços registrados, além de existirem muitas referências inválidas para arquivos WSDL. Contudo, são as melhores opções para localizar \textit{Web Services}, uma vez que motores de busca não proveem mecanismos para localizar apenas este tipo de serviço.

\subsubsection{Metodologia para um Motor de Busca de WS Escalável}

Segundo \cite{lausen2007finding}, trabalhos prévios na área de \textit{Web Services} Semânticos focavam, principalmente, em prover meios para descrever funcionalidades de \textit{Web Services} de uma forma mais acurada, para permitir uma linguagem muito expressiva para busca de serviços. No entanto, nenhum dos trabalhos discute em detalhes como estas descrições são criadas. Assim, \cite{lausen2007finding} desenvolveu um motor de busca escalável de \textit{Web Services}. O motor possui as seguintes características:

\begin{itemize}
	\item varre a Web em busca de documentos WSDL, que apesar de ser um documento técnico, contém muitas informações textuais. Ele analisa os WSDL, especialmente as características ao redor do \textit{endpoint}.
  \item correlaciona informações de cada serviço. Por exemplo, se um host particular tem mais que um serviço, isto aumenta as chances de ele ser um provedor profissional e provavelmente mais relevente que outros. Outras informações como a disponibilidade e o tempo de resposta de um \textit{endpoint} específico podem ser utilizadas como critério de classificação. Além disto, informações como o IP do serviço podem ser utilizadas para deduzir a localização do mesmo. Com uma análise similar, é possível gerar anotações semânticas sobre a natureza do serviço (ex. se ele é comercial ou acadêmico, baseado na organização proprietário de uma determinada faixa de IPs);
  \item análise informações implícitas e explícitas do comportamento do usuário, enquanto interagindo com o motor de busca.
\end{itemize}

Desta forma, é possível gerar, automaticamente, modelos iniciais para um serviço. Estes podem ser refinados usando as informações implícitas e explícitas obtidas do comportamento do usuário. Diferente de outras propostas semânticas que tem a premissa de seleção automática de serviços, os autores desta proposta acreditam em seleção manual, feita por pessoas.

\subsubsection{Implementação}

O motor de busca desenvolvido em \cite{lausen2007finding}, em experimentos iniciais, permitiu obter mais de 8000 serviços. Dos quais, para muitos pode-se criar descrições indo muito além das informações existentes no WSDL, usando os características citadas na seção anterior.

Para realização do \textit{crawling} foi usado o \textit{Heritrix Web crawler}\footnote{\url{http://crawler.archive.org}}, adicionando-se regras de escopo especiais.
No processo de análise dos resultados, são removidos os WSDL's duplicados e averiguado se o \textit{end point} aponta para um endereço válido. Além disso, informações de tempo de resposta e se o \textit{end point} está adequado a versão do protocolo SOAP que ele declara implementar. Um critério de relevância para o serviço foi o número de páginas que apontam para o serviço.

\subsubsection{Conclusões}

Segundo \cite{lausen2007finding}, modelos padrões de motores de busca sintáticos como o \textit{Google}, sozinho, não são bem adequados para descoberta de serviços. Propostas que têm focado em processos completamente autônomos, baseados em interface máquina-a-máquina, tais como UDDI ou outros trabalhos na área de \textit{Web Services} Semânticos, não são adequados para um ambiente heterogêneo e aberto como a Web. A proposta realiza um pós-processamento dos documentos WSDL e constrói modelos semânticos dos serviços, aplicando análise heurística, verificada pela comunidade. \cite{lausen2007finding} declara que outras soluções não alcançam a ampla cobertura junto com uma relativa acurácia na obtenção dos resultados, providas pela solução desenvolvida.
