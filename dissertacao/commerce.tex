\section{Comércio Eletrônico: \textit{E-Commerce}, \textit{M-Commerce} e \textit{T-Commerce}}

No início da \textit{World Wide Web} (comumente denominada apenas \textit{Web}) \nomenclature{Web, WWW}{\textit{World Wide Web}}
as primeiras páginas de \textit{Internet} possuíam apenas conteúdo estático permitindo pouca ou nenhuma interação
do usuário. A \textit{Web} era apenas um repositório de informações. Com a crescente demanda de troca
de informações entre empresas, o advento de padrões abertos de comunicação, 
e a necessidade das mesmas de alcançarem novos clientes e mercados, durante a década de 90 
começou a surgir um novo gênero de \textit{Web sites}: os \textit{sites} de comércio eletrônico\cite{chu2007evolution}.

Tais \textit{Web sites} possibilitaram a realização de negócios entre compradores e vendedores
e entre vendedores e seus parceiros. Desta forma surgiu o \textit{E-Commerce}. Este é baseado na possibilidade
de realização de transações de forma eletrônica. Segundo \cite{veijalainen2006transaction}:

\begin{quote} 
"uma transação eletrônica
é uma venda ou compra de produtos ou serviços, entre empresas, familiares, indivíduos, governos e outras organizações
públicas ou privadas, conduzidas por redes mediadas por computador."
\end{quote}

A troca de informações entre essas empresas era feita por meio de \textit{Eletronic Data Interchange} (EDI)\nomenclature{EDI}{\textit{Eletronic Data Interchange}} (troca eletrônica de informações), no entanto, isto requeria realização de acordos entre as organizações\cite{chu2007evolution},
o que poderia dificultar tais parcerias. O advento de padrões abertos como o XML permitiu a evolução
de tais processos de intercâmbio de dados e integração entre empresas.

Atualmente existem diversas empresas consolidadas na área de comércio eletrônico. Algumas nasceram na era digital
e vendem exclusivamente pela \textit{Internet}, alcançando
novos clientes e mercados nacionais e internacionais.

Com a recente popularização de dispositivos móveis e da \textit{Internet} sem fio como rede de grande abrangência (por exemplo, utilizando as tecnologias
de comunicação móvel de 3ª geração como o \textit{Universal Mobile Telecommunications System} - UMTS)\nomenclature{UMTS}{\textit{Universal Mobile Telecommunications System}}, surgem novas oportunidades para as lojas de comércio eletrônico. 
Estas entram em uma nova era, com novas perspectivas de captação de clientes e mercados. Com isto surgem novas tendências 
como o denominado Comércio Móvel (\textit{M-Commerce})\nomenclature{M-Commerce}{Comércio Móvel}, onde as transações são feitas utilizando-se dispositivos e redes de acesso móveis, tais como \textit{Wireless Local Area Networks} (WLAN's)\nomenclature{WLAN}{\textit{Wireless Local Area Network}}\nomenclature{LAN}{\textit{Local Area Network}}, redes de telecomunicações 2G ou 3G, conexões \textit{Bluetooth} ou infravermelho\cite{veijalainen2006transaction}.

Tais dispositivos móveis permitem que os usuários possam realizar compras em qualquer lugar que eles estejam,
até mesmo em suas horas livres, durante o trajeto para o trabalho, ou no horário de almoço. Desta forma,
as lojas virtuais têm um mercado em potencial para aumentar suas vendas.

Com o advento dos sistemas de televisão digital interativa e a possibilidade de se ter 
aplicativos executando juntamente com a programação áudio-visual, abre-se um
novo mercado para as lojas de comércio eletrônico: a venda de produtos pela TV.
Tais serviços de vendas pela TV não são novos, mas antes da TV Digital Interativa (TVDi),
os canais de venda pela TV apenas anunciavam produtos e os telespectadores que 
desejavam comprar um produto precisavam utilizar outro canal de comunicação, como
o telefone ou recorrer ao \textit{Web site} da loja na \textit{Internet}.
Utilizando-se os recursos da TV Digital Interativa, todo o processo de compra
pode ser realizado diretamente pelo controle remoto da TV, desde que a mesma
esteja conectada à \textit{Internet}.

As tecnologias da TV Digital trazem um novo benefício aos usuários: a possibilidade
de comprar produtos a partir de um receptor de TV Digital, caracterizando uma nova modalidade de comércio
eletrônico, denominada \textit{T-Commerce}.

Em países norte-americanos e europeus, que têm sistemas de TV Digital Interativa
há mais tempo que o Brasil, existem algumas empresas disponibilizando soluções para a área de \textit{T-Commerce}
\footnote{\url{http://www.ensequence.com/t-commerce}}
\footnote{\url{http://www.digisoft.tv/applications.html}}
\footnote{\url{http://www.icuetv.com/ets\_platform/applications/t\_commerce}}.
No Brasil, apesar de as transmissões de TV Digital terem iniciado somente em 2007 
na cidade de São Paulo\cite{inicio-transmissao-tv-digital-saopaulo}, em 2010
já \replaced{há}{haviam} soluções iniciais para \textit{T-Commerce}\cite{extra-vendas-tvd}, apesar de a finalização
da compra ainda precisar ser feita utilizando-se outros canais como o telefone.

Devido ao Brasil ser um país de características bem diferentes dos outros países, 
espera-se que a interatividade seja um grande diferencial para o país.
Como o Governo Federal pretende 
realizar inclusão social/digital por meio da TV, de acordo com o Decreto número 4.901, de 26 de novembro de 2003,
espera-se que a disponibilização de serviços pela TV seja crescente.
Um fator importante para a difusão da interatividade no país é a existência 
de aparelhos de TV em 96\% das residências\cite{ibge-pnad}, tendo grandes
possibilidades de as aplicações interativas terem um enorme público, considerando
ainda as dimensões continentais do Brasil e sua enorme população de 185.712.713 habitantes, segundo
o Censo 2010\cite{censo2010}.

Outra medida tomada pelo governo que beneficiará a interatividade pela TV Digital é a criação do 
Plano Nacional de Banda Larga (PNBL)\nomenclature{PNBL}{Plano Nacional de Banda Larga}, por meio do Decreto número 7.175, de 12 de maio de 2010,
visando prover \textit{Internet} banda larga, a um custo reduzido, em municípios onde não exista tal serviço.


