\section{Justificativa} \label{sec:justificativa}

Como o início das operações do Sistema Brasileiro de TV Digital (SBTVD) é bastante recente, onde a primeira transmissão
de sinal digital foi realizada em 2007 na cidade de São Paulo, 
até a data de elaboração da presente dissertação, só conhecia-se 
um caso de aplicação de \textit{T-Commerce} no Brasil, como citado em \cite{extra-vendas-tvd},
no qual \deleted{uma aplicação }apenas \replaced{exibem-se}{exibe} os produtos e o processo de compra deve ser realizado pelo usuário utilizando
outro canal como a loja virtual (\textit{Internet}) ou o \textit{call center} da empresa.

Além disso, tal aplicação de comércio eletrônico foi desenvolvida para 
o \textit{middleware} \textit{ByYou} (implementação do Ginga desenvolvida pela empresa TOTVS)\cite{totvs-byyou},
e seu uso fica restrito aos conversores digitais e \replaced{televisores}{TV's} que possuam
tal \textit{middleware}. Por usar API's específicas do \textit{ByYou}, a aplicação só executa em tal implementação 
do \textit{middleware} Ginga.

Como o Brasil possui diversos casos de sucesso no mercado de comércio eletrônico em larga escala, 
a disponibilização de aplicações de comércio para TV Digital é uma tendência natural,
podendo aumentar consideravelmente as vendas das empresas do ramo, devido à grande penetração
do aparelho de TV nos lares brasileiros (cerca de 96\%)\cite{ibge-pnad}, além 
de ser uma nova plataforma de \textit{E-Commerce} para os usuários.

Da falta de uma arquitetura para provimento de comércio eletrônico para o SBTVD, surgiu este trabalho
de dissertação, que apresenta uma proposta de arquitetura orientada a serviços (\textit{Service Oriented Architecture} - SOA) para a estruturação
de um ambiente de \textit{T-Commerce}. Considerando que tal arquitetura é bastante
utilizada para a integração de sistemas heterogêneos, ele vai ao encontro de um dos objetivos
do projeto: prover uma arquitetura de \textit{T-Commerce} que utilize serviços \textit{Web}
que, juntos, \deleted{contemplem tanto as exigências e padrões de qualidade das empresas quanto as necessidades e desejos dos usuários}
\added{atuem de forma interoperável com o SBTVD e os padrões W3C}.

Desta forma, como a arquitetura SOA pode, comumente, ser baseada em \textit{Web Services} SOAP, faz-se necessária
a implementação de protocolos como \textit{Hypertext Transfer Protocol} (HTTP) e 
\textit{Simple Object Access Protocol} (SOAP) em que se baseiam as tecnologias relacionadas a SOA,
sendo tais implementações \deleted{um dos }objetivos principais deste trabalho.

Com a implementação dos protocolos HTTP e SOAP (onde não se tem conhecimento, até a data de elaboração
desta dissertação, de nenhuma implementação de código aberto dos mesmos para o ambiente de TVD) criam-se enormes possibilidades
de construção de aplicações para integração entre \textit{Web} e TV, um dos objetivos deste projeto
com sua arquitetura de \textit{T-Commerce}. Como durante as pesquisas observou-se que a falta
de implementação de tais protocolos era uma queixa recorrente nos fóruns da Comunidade
Ginga no Portal do Software Público\footnote{\url{http://www.softwarepublico.gov.br/dotlrn/clubs/ginga/}}
e em outros fóruns de discussão, a implementação de tais protocolos representa uma grande contribuição
para o desenvolvimento do SBTVD. Assim, após a publicação inicial das implementações
dos protocolos HTTP e SOAP, alguns trabalhos importantes foram desenvolvidos, como será apresentado no Capítulo \ref{cap:ncluasoap}.
