\subsection{\textit{A New Description Model of Web Service} \cite{wu2009new}} \label{sec:ws-description-model}

Como descrever e descobrir serviços é um problema chave de integração de aplicações baseadas em \textit{Web Services}. Em \cite{wu2009new} é desenvolvido um modelo para descrição semântica de \textit{Web Services}, utilizando padrões como WSDL e OWL-S\nomenclature{OWL-S}{\textit{Semantic Markup for Web Services}} (\textit{Semantic Markup for Web Services}), estendendo a arquitetura de UDDI para procurar e combinar \textit{Web Services} mais fácil e acuradamente.

As descrições podem ser classificadas em descrição de objetos, de interface, de realização e descrições não-funcionais (como parâmetros de QoS\nomenclature{QoS}{\textit{Quality of Service}}).

O principal método para descrever \textit{Web Services} é a descrição baseada em WSDL, que é usada para descrever o formato da mensagem, tipos de dados, operações, \textit{binding} e outros. No entanto, WSDL é usado para descrever informações básicas de \textit{Web Services}, ele não possui recursos para prover descrições semânticas do serviço.

Para prover descrições semânticas, a proposta de \cite{wu2009new} introduz o uso de OWL-S, um tipo de ontologia de \textit{Web Service}, definida com uso da linguagem OWL\nomenclature{OWL}{\textit{Web Ontology Language}}, a \textit{Web Ontology Language}. A OWL-S provê um conjunto de linguagens padrões para provedores analisarem e descreverem a performance e atributos de \textit{Web Services}, por meio de programas de computador. Assim, \textit{Web Services} podem ser localizados, associados, combinados e monitorados automaticamente. A OWL-S define três principais aspectos de informações semânticas: perfil, modelo e bases de serviço. Perfil de serviço descreve o que o \textit{Web Service} faz, incluindo informações básicas. Modelo de serviço descreve como o \textit{Web Service} opera, e bases de serviço descrevem os detalhes de acesso ao \textit{Web Service}, incluindo: protocolo de rede, formato de mensagem e outros detalhes. Perfil de serviços OWL-S proveem descrição semântica de funcionalidades do serviço, incluindo entrada e saída de dados. Perfil de serviço também atribui parâmetros para qualidade de serviço, tais como garantia de qualidade, \textit{rank} de qualidade e tudo mais, mas não provê definições detalhadas de atributos. Portanto, a proposta de \cite{wu2009new} inclue um modelo detalhado de QoS para prover medidas concretas de qualidade de serviço.

Métodos existentes de descrição de \textit{Web Services} têm dois aspectos de desvantagem: devido à insuficiência de informações semânticas e à dependência de associação de palavras-chave, a acurácia de localização de \textit{Web Services} é baixa. Por outro lado, isto afeta a qualidade de serviço devido a falta de descrições de QoS.


