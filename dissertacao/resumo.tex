\chapter{Resumo}

% Acesso ao comando \MyTitle devido ao uso do pacote authoraftertitle
\section*{\MyTitle} % * é pra não numerar a seção

\replaced{Esta}{A presente} dissertação descreve uma arquitetura orientada a serviços para provimento de comércio eletrônico
pela TV Digital, por meio do Sistema Brasileiro de TV Digital (SBTVD),
desenvolvida para o sub-sistema Ginga-NCL do \textit{middleware} Ginga.
A arquitetura proposta utiliza serviços de diferentes provedores (nas áreas de telecomunicações, logística e outros) para compor 
uma estrutura de \textit{T-Commerce}. Tais serviços são desenvolvidos
considerando aspectos de interoperabilidade, utilizando o protocolo SOAP,
para o qual é apresentada uma implementação, juntamente com o
HTTP, como base para o desenvolvimento de toda
a arquitetura e um dos objetivos principais do projeto.

Com a arquitetura elaborada, uma aplicação cliente, desenvolvida
em NCL e Lua, é apresentada como prova de conceito do uso das
implementações dos protocolos e da arquitetura proposta.
Tal aplicação utiliza o \textit{framework} LuaOnTV para a construção
da interface gráfica de usuário para a TV Digital, o qual foi estendido
neste trabalho, com as melhorias sendo apresentadas ao longo do mesmo.

O trabalho ainda apresenta um conjunto de aplicações desenvolvidas a partir
dos \textit{frameworks} construídos, que complementam as funcionalidades da aplicação
de T-Commerce, como leitor de RSS e rastreamento de encomendas.

A partir do ambiente de desenvolvimento montado para a construção das aplicações, 
contendo a implementação de referência do sub-sistema Ginga-NCL do \textit{middleware} Ginga, 
nativamente instalada, foi gerada uma distribuição Linux que permite que tal ambiente
seja instalado em qualquer computador ou máquina virtual, para permitir o desenvolvimento
de arquitetura semelhante ou extensão da arquitetura proposta.



\textbf{Palavras-chave:} SBTVD, Ginga, Ginga-NCL, \textit{Web Services}, HTTP, SOAP, SOA, Lua, NCL, NCLua, NCLua HTTP, NCLua SOAP, LuaOnTV 2.0, 
\textit{E-Commerce}, \textit{T-Commerce}

\chapter{Abstract}

\section*{Service-oriented architecture for electronic commerce in the Brazilian Digital Television System}

\replaced{This}{The current} dissertation describes a service-oriented architecture for providing of digital TV electronic commerce, 
through the Brazilian Digital Television System,
developed to the Ginga-NCL sub-system of the Brazilian Ginga middleware.
The proposed architecture uses services from distinct providers
(at telecommunication, logistics and other areas) to compose a T-Commerce structure. 
Such services are developed considering interoperability aspects,
using the SOAP protocol, for wich is presented
an implementation, together with the HTTP protocol, as a basis
for the development of the entire architecture and 
one of the project main goals.

With the \replaced{architecture designed}{elaborated architecture}, a client application, developed
in NCL and Lua languages, is presented \replaced{as}{how} a proof of concept
of the protocols implementations and proposed architecture use.
Such application uses the LuaOnTV framework to build
a Digital TV graphical user interface, wich was extended in this dissertation, with the improvements being presented along it.

The work also presents a set of applications developed from
the constructed frameworks that complement the T-Commerce application functionalities, 
such as RSS reader and orders tracking.

From the mounted development environment for applications building,
containing the reference implementation of the  Ginga-NCL sub-system of the Ginga middleware,
natively installed, a Linux distribution was generated that enables such environment
to be installed on any computer or virtual machine, to allow the development
of similar architecture or extension of the proposed \replaced{one}{architecture}.


\textbf{Keywords:} ISDB-TB, Ginga, Ginga-NCL, Web Services, HTTP, SOAP, SOA, Lua, NCL, NCLua, NCLua HTTP, NCLua SOAP, LuaOnTV 2.0, 
E-Commerce, T-Commerce
