\section{Objetivos} \label{sec:objetivos}

\subsection{Geral}

Propor e desenvolver uma arquitetura orientada a serviços,
por meio de \textit{Web Services} SOAP, para provimento de comércio eletrônico
pela TV Digital, favorecendo a convergência \textit{Web}-TV. 

\subsection{Específicos}

Como objetivos específicos, \replaced{propõe-se}{pretende-se}:
\begin{enumerate}[A)]
  \item \added{elicitar requisitos funcionais e não funcionais a serem atendidos por} \deleted{propor} 
  uma arquitetura baseada em padrões de serviços \textit{Web}, a ser utilizada para comércio eletrônico via TV Digital;
  
  \item \added{propor uma arquitetura de serviços de \textit{T-Commerce}, que atenda aos requisitos citados;}
  
	\item a partir da arquitetura \deleted{acima}, implementar um \textit{framework} para comunicação de dados, baseado nos protocolos HTTP e SOAP,
	para permitir a comunicação de aplicações de TV Digital, desenvolvidas nas linguagens NCL e Lua,
	que permita a interoperação com serviços \textit{Web};
	
  \item caracterizar o emprego do \textit{framework} de comunicação de dados para desenvolvimento de aplicações
  tais como Leitor de RSS, Rastreador de Encomendas, Cliente de Twitter, \added{Enquete e \textit{Quiz}};
	
	\item estender o \textit{framework} LuaOnTV\cite{junior2009luacomp} incluindo recursos
	de temas para permitir a definição centralizada das características visuais das aplicações
	desenvolvidas, além de incluir suporte a múltiplos dispositivos com diferentes resoluções
	de tela, como TV's e aparelhos celulares;
	
	\item \deleted{propor }um modelo de desenvolvimento de aplicações para TV Digital (TVD), de forma 
  que todos os formulários da aplicação (páginas/telas) tenham um mesmo
  conjunto de componentes básicos, permitindo a criação de \textit{templates} para definir a identidade
  visual da mesma. Assim, ao ser alterado o \textit{template}, toda
  a identidade visual da aplicação deverá ser alterada;
  
  \deleted{	
	%\item 
	montar uma distribuição Linux contendo a implementação de referência do sub-sistema Ginga-NCL
	do \textit{middleware} Ginga, executando de forma nativa (não virtualizada) no sistema operacional, já
	contendo todas as ferramentas necessárias e configuradas para o desenvolvimento de aplicações
	de TV Digital em NCL e Lua, permitindo o funcionamento como \textit{LiveCD} (sem necessidade de instalação
	da distribuição) ou instalando a mesma em um computador, à semelhança do realizado em \cite{soset}.
	}
\end{enumerate}


