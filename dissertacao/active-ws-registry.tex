\subsection{\textit{Active Web Service Registries} \cite{treiber2007active}} \label{sec:active-ws-reg}

A proposta para descoberta de serviços, apresentada em \cite{treiber2007active}, consiste em uma arquitetura distribuída para repositórios de descoberta de serviços, denominada \textit{Active Web Service Registry} (AWSR)\nomenclature{AWSR}{\textit{Active Web Service Registry}}. A arquitetura utiliza o formato \textit{Atom} de sindicalização de notícias para registrar os serviços diponibilizados por um provedor. Desta forma, a publicação dos serviços pode ser feita utilizando um dos muitos softwares para geração de \textit{feeds} RSS e \textit{Atom}, disponíveis no mercado.

É utilizado o modelo de dados \textit{Atom} para armazenar informações relevantes sobre os serviços, como descrições textuais, exemplos de utilização e o endereço do WSDL que descreve o serviço. Desta forma, os clientes podem receber notificações sobre atualizações e novos serviços, utilizando clientes RSS/\textit{Atom} padrões\footnote{Existem muitos softwares que suportam o formato RSS/\textit{Atom}, como clientes de \textit{e-mail}, navegadores \textit{Web}, \textit{widgets} e softwares específicos.}.

Na arquitetura proposta, cada provedor de serviços é responsável por gerar seu \textit{feed Atom}, utilizando algum software padrão, para publicar atualizações e novos serviços, criando um ambiente descentralizado, diferente dos modelos tradicionais de UDDI. Desta forma, cada provedor é um \textit{Active Web Service Registry}. O processo de descoberta é feito pelo cliente, que assina este \textit{feed}, utilizando algum software padrão.

Para que o cliente possa descobrir os \textit{Active Web Service Registry}, existem duas propostas citadas em \cite{treiber2007active}: 
\begin{itemize}
	\item adicionar uma tag \textit{link} dentro do cabeçalho de uma página Web, que, quando acessada pela maioria dos navegadores Web atuais, avisará o usuário que aquela página possui um \textit{feed} e solicitará a assinatura do mesmo; e
  \item usar sites indexadores, similares a agregadores de notícias RSS/\textit{Atom} (como o \textit{FeedBurner}\footnote{\url{http://feedburner.google.com}}), para publicar os \textit{feeds} AWSR.
\end{itemize}

A arquitetura é interessante, pois utiliza um modelo de dados \textit{Atom}, amplamente difundido, permitindo a utilização de ferramentas existentes no mercado, tanto para gerar \textit{feeds} para registrar os serviços, quanto para manter os usuários a par de atualizações e novos serviços disponibilizados. Além de criar um ambiente distribuído para a publicação de serviços.

