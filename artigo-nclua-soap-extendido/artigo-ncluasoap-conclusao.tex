Os módulos NCLua HTTP e NCLua SOAP facilitam a convergência entre \textit{Web} e TV, escondendo
os detalhes de implementação dos protocolos HTTP e SOAP do desenvolvedor
de aplicações de TVDi, permitindo o surgimento de novas aplicações interativas 
que fazem uso de conteúdo da \textit{Internet}.
O NCLua SOAP permitiu o desenvolvimento de aplicações de \textit{T-Government}
como apresentado em \cite{tgov2010barbosa}, sistema de recomendação\cite{gatto2010BIPODiTVR} entre outras. 
Na página do projeto, em http://ncluasoap.manoelcampos.com,
existe um \textit{link} para o fórum de discussão do módulo, onde alguns usuários relatam a
utilização do mesmo, por exemplo, em aplicações de \textit{T-Learning}.

Atualmente, o processo de obtenção dos dados para realizar a chamada a um método remoto em um \textit{Web Service}
ainda é praticamente todo manual, no entanto, 
após o desenvolvedor obter tais dados para o método remoto desejado, a realização da requisição é bastante simplificada,
principalmente pelo fato de Lua ser uma linguagem de tipagem dinâmica \cite{ierusalimschy2006programming}, 
não obrigando a declaração de variáveis com seus respectivos tipos.
O \textit{feedback} dos usuários tem mostrado que o módulo está sendo bastante útil para 
a comunidade, além de permitir a evolução do mesmo.
